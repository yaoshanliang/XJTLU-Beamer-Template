\documentclass{libs/XJTLU_format}
% Inserting the preamble file with the packages
%%%%%%%%%%%%%%%%%%%%%%%%%%%%%%%%%%%%%%%%%%%%%%%%%%%%%%%%%%%%%%%%%%%%%
%% This file contains the packages that can be used in the beamer. %%
%%%%%%%%%%%%%%%%%%%%%%%%%%%%%%%%%%%%%%%%%%%%%%%%%%%%%%%%%%%%%%%%%%%%%
% Package to fonts family
\usepackage[T1]{fontenc}
% Package to accentuation
\usepackage[utf8]{inputenc}
% Package to Figures
\usepackage{graphicx}
% Package to the colors
\usepackage{color}
% Package to the colors
\usepackage{xcolor}
% Packages to math symbols and expressions
\usepackage{amsfonts, amssymb, amsmath}
% Package to multiple lines and columns in table
\usepackage{multirow, array} 
% Package to create pseudo-code
% For more detail of this package: http://linorg.usp.br/CTAN/macros/latex/contrib/algorithm2e/doc/algorithm2e.pdf
\usepackage{algorithm2e}
% Package to insert code
\usepackage{listings} 
\usepackage{keyval}
% Package to justify text
\usepackage[document]{ragged2e}
% Package to manage the bibliography
\usepackage[backend=biber, style=numeric, sorting=none]{biblatex}
% Package to facilities quotations
\usepackage{csquotes}
% Package to use multicols
\usepackage{multicol}
\usepackage{transparent}

% Inserting the references file
\bibliography{references.bib}

% Title
\title[XJTLU Beamer Template]{\huge\textbf{XJTLU Beamer Template}}
% Subtitle
\subtitle{Creating Presentations}
% Author of the presentation
\author{Author}
% Institute's Name
\institute[XJTLU]{
    % email for contact
    \normalsize{\email{author@xjtlu.edu.cn}}
    \newline
    % Department Name
    \department{School of Advanced Technology}
    \newline
    % University name
    \university{Xi’an Jiaotong-Liverpool University}
}
% date of the presentation
\date{\today}


%%%%%%%%%%%%%%%%%%%%%%%%%%%%%%%%%%%%%%%%%%%%%%%%%%%%%%%%%%%%%%%%%%%%%%%%%%%%%%%%%%
%% Start Document of the Presentation                                           %%               
%%%%%%%%%%%%%%%%%%%%%%%%%%%%%%%%%%%%%%%%%%%%%%%%%%%%%%%%%%%%%%%%%%%%%%%%%%%%%%%%%%
\begin{document}
% insert the code style
%%%%%%%%%%%%%%%%%%%%%%%%%%%%%%%%%%%%%%%%%%%%%%%%%%%%%%%%%%%%%%%%%%%%%%%%%%%%%%%%%%%
%% This file contains the style of the codes show in slides.                     %%
%% The package used is listings, but it possible to used others.                 %%
%%%%%%%%%%%%%%%%%%%%%%%%%%%%%%%%%%%%%%%%%%%%%%%%%%%%%%%%%%%%%%%%%%%%%%%%%%%%%%%%%%%

% color used in the code style
\definecolor{codegreen}{rgb}{0,0.6,0}
\definecolor{codegray}{rgb}{0.5,0.5,0.5}
\definecolor{codepurple}{rgb}{0.58,0,0.82}
\definecolor{codebackground}{rgb}{0.95,0.95,0.92}

% style of the code!
\lstdefinestyle{codestyle}{
    backgroundcolor=\color{codebackground},   
    commentstyle=\color{codegreen},
    keywordstyle=\color{magenta},
    numberstyle=\tiny\color{codegray},
    stringstyle=\color{codepurple},
    basicstyle=\ttfamily\footnotesize,
    frame=single,
    breakatwhitespace=false,         
    breaklines=true,                 
    captionpos=b,                    
    keepspaces=true,                 
    numbers=left,                    
    numbersep=5pt,                  
    showspaces=false,                
    showstringspaces=false,
    showtabs=false,                  
    tabsize=2,
    title=\lstname 
}

\lstset{style=codestyle}


%% ---------------------------------------------------------------------------
% First frame (with tile, subtitle, ...)
\begin{frame}{}
    \maketitle
\end{frame}

%% ---------------------------------------------------------------------------
% Second frame
\begin{frame}{Table of Contents}
    \begin{multicols}{2}
        \tableofcontents
    \end{multicols}
\end{frame}

%% ---------------------------------------------------------------------------
% This presentation is separated by sections and subsections
\section{Introduction}

\begin{frame}
    \begin{itemize}[<+-| alert@+>] % 当然,除了alert,手动在里面插 \pause 也行
    	\item This is a slide template created by latex for XJTLUers.
        \item Overleaf  \\ \url{https://www.overleaf.com/latex/templates/thu-beamer-theme/vwnqmzndvwyb}
        \item GitHub \\ \url{https://github.com/yaoshanliang/XJTLU-Beamer-Theme}
    \end{itemize}
    
    \note {Write your notes.\\}
    \begin{note}
        {Write your notes here}
    \end{note}
\end{frame}

\section{Literature Review}

\subsection{Usage}

\begin{frame}{Usage}
    \begin{itemize}
        \item Beamer is a powerful and flexible \LaTeX{} class to create great looking presentations.   \url{https://www.overleaf.com/learn/latex/Beamer}
        \item Modify from Template Beamer UFC \cite{origin}
    \end{itemize}
\end{frame}

\subsection{Features}

\begin{frame}{Features}
	The visual design follows \href{https://www.xjtlu.edu.cn/en/about/professional-services/centre-for-knowledge-and-information/university-marketing-and-communications/visual-identity-assets}{\emph{VISUAL IDENTITY ASSETS}} from XJTLU.
    \begin{enumerate}
        \item {XJTLU \example{NAVY} (RGB: 1, 54, 68)}
        \item {XJTLU \example{PURPLE} (RGB: 206, 87, 193)}
    \end{enumerate}
\end{frame}


\section{Methodology}

\subsection{Blocks}
\begin{frame}{Blocks}
    % Blocks styles
    \begin{block}{Block I}
        Text
    \end{block}

    \begin{alertblock}{Block II}
        Text
    \end{alertblock}

    \begin{exampleblock}{Block III}
        Text
    \end{exampleblock}  
    
    \successbox{Success box}
    \alertbox{Alert box}
    \simplebox{Simple box}
 
\end{frame}


%% ---------------------------------------------------------------------------
\subsection{Alorgithms}
\begin{frame}{Algorithms (pseudocode)}
    \begin{algorithm}[H]
        \SetAlgoLined
        \LinesNumbered
        \SetKwInOut{Input}{input}
        \SetKwInOut{Output}{output}
        \Input{x: float, y: float}
        \Output{r: float}
        \While{True}{
          r = x + y\;
          \eIf{r >= 30}{
           ``O valor de $r$ é maior ou iqual a 10.''\;
           break\;
           }{
           ``O valor de $r$ = '', r\;
          }
         } 
         \caption{Algorithm Example}
    \end{algorithm}
\end{frame}

\begin{frame}{Algorithms}
    \lstset{language=Python}
    \lstinputlisting[language=Python]{code/main.py}
\end{frame}

\subsection{Equations}

\begin{frame}{Equation}
    \begin{block}{Equation without numbers} 
        \begin{equation*}
            J(\theta) = \mathbb{E}_{\pi_\theta}[G_t] = \sum_{s\in\mathcal{S}} d^\pi (s)V^\pi(s)=\sum_{s\in\mathcal{S}} d^\pi(s)\sum_{a\in\mathcal{A}}\pi_\theta(a|s)Q^\pi(s,a)
        \end{equation*}
    \end{block}
%    \begin{exampleblock}{Multiple equations\footnote{If containing text in equations,use $\backslash$mathrm\{\} or $\backslash$text\{\}}}
%       
%        \begin{align}
%            Q_\mathrm{target}&=r+\gamma Q^\pi(s^\prime, \pi_\theta(s^\prime)+\epsilon)\\
%            \epsilon&\sim\mathrm{clip}(\mathcal{N}(0, \sigma), -c, c)\nonumber
%        \end{align}
%    \end{exampleblock}
\end{frame}

\begin{frame}
    \begin{block}{Equation with numbers}
        % Taken from Mathmode.tex
        \begin{multline}
            A=\lim_{n\rightarrow\infty}\Delta x\left(a^{2}+\left(a^{2}+2a\Delta x+\left(\Delta x\right)^{2}\right)\right.\label{eq:reset}\\
            +\left(a^{2}+2\cdot2a\Delta x+2^{2}\left(\Delta x\right)^{2}\right)\\
            +\left(a^{2}+2\cdot3a\Delta x+3^{2}\left(\Delta x\right)^{2}\right)\\
            +\ldots\\
            \left.+\left(a^{2}+2\cdot(n-1)a\Delta x+(n-1)^{2}\left(\Delta x\right)^{2}\right)\right)\\
            =\frac{1}{3}\left(b^{3}-a^{3}\right)
        \end{multline}
    \end{block}
\end{frame}


%\begin{frame}[fragile]{\LaTeX{} Commands}
%    \begin{exampleblock}{Commands}
%        \centering
%        \footnotesize
%        \begin{tabular}{llll}
%            \cmd{chapter} & \cmd{section} & \cmd{subsection} & \cmd{paragraph} \\
%            Chapter & Section & Subsection & Paragraph \\\hline
%            \cmd{centering} & \cmd{emph} & \cmd{verb} & \cmd{url} \\
%            Centre Align & Emphasis & Verbatim & Hyperlink \\\hline
%            \cmd{footnote} & \cmd{item} & \cmd{caption} & \cmd{includegraphics} \\
%            Foodnote & Item & Caption & FigP\&Pic \\\hline
%            \cmd{label} & \cmd{cite} & \cmd{ref} \\
%            Label & Citing & Referring\\\hline
%        \end{tabular}
%    \end{exampleblock}
%    \begin{exampleblock}{Environment Command}
%        \centering
%        \footnotesize
%        \begin{tabular}{lll}
%            \env{table} & \env{figure} & \env{equation}\\
%            Table & Figure & Equation \\\hline
%            \env{itemize} & \env{enumerate} & \env{description}\\
%            Bullets & Numbering & Description \\\hline
%        \end{tabular}
%    \end{exampleblock}
%\end{frame}


\subsection{Figures}
\begin{frame}{Figures}
    \begin{figure}
        \centering
        \caption{Logo of XJTLU.}
        \includegraphics[scale=0.5]{libs/XJTLU_logo.png}
        \source{Description of XJTLU}
        \label{fig:XJTLU_logo}
    \end{figure}
\end{frame}

\subsection{Tables}
\begin{frame}{Tables}
\begin{table}[hbt]
  \begin{tabular}{l|cc}
    1& 2& \\
    \hline
    3& 4& \\
    5& 6&
  \end{tabular}
  \caption{}
\end{table}

\end{frame}

%% ---------------------------------------------------------------------------
% This frame show an example to insert multi-columns
\subsection{Multi-columns}
\begin{frame}{Multi-columns}
    \begin{columns}{}
        \begin{column}{0.5\textwidth}
            \justify
            É possível colocar mais de uma coluna utilizando os comandos de $\backslash$begin\{column\}\{\} e $\backslash$end\{column\}
        \end{column}
        \begin{column}{0.5\textwidth}
            \justify
            Porém, o espaçamento deve ser proporcional entre as colunas para que estas colunas não entrem em coflito. O espaçamento é dado pelo segundo argumento do $\backslash$begin.
        \end{column}
    \end{columns}   
    \begin{columns}{}
        \begin{column}{0.5\textwidth}
            \justify
            É possível colocar mais de uma coluna utilizando os comandos de $\backslash$begin\{column\}\{\} e $\backslash$end\{column\}
        \end{column}
        \begin{column}{0.5\textwidth}
            \justify
            Porém, o espaçamento deve ser proporcional entre as colunas para que estas colunas não entrem em coflito. O espaçamento é dado pelo segundo argumento do $\backslash$begin.
        \end{column}
    \end{columns}     
\end{frame}


%% ---------------------------------------------------------------------------
% Reference frames
\begin{frame}[allowframebreaks]
    \frametitle{Reference}
    \printbibliography
\end{frame}

%% ---------------------------------------------------------------------------
% Final frame
\begin{frame}{}
    \centering
    \huge{\textbf{\example{Thank You !}}}
    
    
\end{frame}

\end{document}